%Capa
\renewcommand{\sfdefault}{\rmdefault}
\imprimircapa


%Folha de rosto sem número de página
\setcounter{page}{2}
\renewcommand{\sfdefault}{\rmdefault}
\imprimirfolhaderosto*


% Ficha Catalográfica
\renewcommand{\sfdefault}{\rmdefault}
\begin{fichacatalografica}
    % \includepdf{}
\end{fichacatalografica}

% Folha de aprovação
% \includepdf[pagecommand={\thispagestyle{empty}}]{}
\cleardoublepage


% Dedicatória
\renewcommand{\sfdefault}{\rmdefault}
\begin{dedicatoria}
    \vspace*{\fill}
    \centering
    \noindent
    \textit{}
    \vspace*{\fill}
\end{dedicatoria}

% Agradecimentos
% \renewcommand{\agradecimentosname}{Agradecimentos}
\renewcommand{\sfdefault}{\rmdefault}
\begin{agradecimentos}[Agradecimentos]
    A ser escrito.
    % Agradeço e dedico este trabalho aos meus pais, Elaine e Luciano, que sempre acreditaram em mim em todas etapas da minha vida. Todos os dias tento fazer o melhor que posso pois tenho grandes exemplos como vocês que sempre fizeram de tudo para garantir que nunca me faltasse nada. Agradeço a minha irmã Mirela pela curiosidade e interesse nas suas perguntas e pela amizade que estamos desenvolvendo ao longo dos anos.

    % Agradeço ao Grupo de Física de Aceleradores do LNLS pela forma incrível com que me receberam. Fundamentalmente tudo que sei sobre a área, aprendi com vocês. Considero uma honra ter diariamente como colegas de trabalho as pessoas que conceberam e concretizaram a ideia do Sirius. Para mim é muito gratificante compartilhar com vocês manhãs, tardes, noites e madrugadas de muito trabalho, discussões e cafés.
    
    % Agradeço à Liu por todas as oportunidades, principalmente a de trabalhar com este tema, e também por sempre estar aberta para discussões e para compartilhar seu conhecimento e experiência. Agradeço ao Fernando por ser um grande companheiro de trabalho desde o início, com sua didática e empolgação para conversar sobre tudo, especialmente ciência. Agradeço ao Ximenes pelas discussões, companhia nas aulas e pela ajuda na organização dos códigos no ínicio deste trabalho.
    
    % Agradeço à Ana Clara por todo companheirismo e amor de todos esses anos. Desde a época dos vestibulares até à pós-graduação, você sempre esteve comigo em cada momento e os fez muito mais felizes. Sua companhia alegra e torna essa jornada mais leve em um nível incomensurável. Sou extremamente grato a você. 
    
    % % Também agradeço a sua família, Paulo (Brêga), Dina e Paulinho por todo carinho com que me acolheram e apoio que nos deram.
    
    % Agradeço ao Prof. Rubens por ter aceitado ser meu orientador e pelas sugestões ao longo do trabalho.
    
    % Agradeço ao programa de pós-graduação do IFGW da Unicamp, aos professores e funcionários que realizam seus trabalhos de forma competente e exemplar, contribuindo enormemente para o desenvolvimento da Física no Brasil.
    
    % Finalmente agradeço à instituição LNLS e aos seus membros, por toda infraestrutura disponibilizada e todo trabalho coletivo de alto nível que resultou no Sirius, o maior complexo científico brasileiro e que possibilitou a existência deste mestrado.
    % \vspace{5mm}
    % {\centering \Large \textbf{Acknowledgements}}
\end{agradecimentos}

% Resumo em Português
\begin{otherlanguage*}{brazil}
\renewcommand{\sfdefault}{\rmdefault}
    \begin{center}{\ABNTEXchapterfont\huge Resumo}\end{center}
    
    Sirius é a nova fonte de luz síncrotron de 4\ts{a} geração e baixa emitância do Laboratório Nacional de Luz Síncrotron (LNLS), onde elétrons de $\SI{3}{\giga\electronvolt}$ são mantidos em condições estáveis, em ultra-alto vácuo ao longo de um anel de armazenamento de $\SI{518}{\meter}$ de circunferência sob a ação de campos eletromagnéticos. A matriz resposta de órbita devido a variações de campos dipolares localizados pode ser usada para ajustar a ótica linear e termos de acoplamento bétatron de uma rede magnética a partir de um modelo, usando o método chamado Linear Optics from Closed Orbits (LOCO). Neste trabalho, o método LOCO foi estudado e implementado no anel de armazenamento do Sirius, a fim de calibrar e corrigir ótica linear e acoplamento durante o comissionamento. Vários testes foram realizados com o código implementado usando dados simulados e medidos, obtendo resultados que verificaram a robustez do método. A escolha do algoritmo de minimização e a inclusão de vínculos nas variações de gradientes nos quadrupolos foram fatores importantes para a aplicação do método no Sirius. O ajuste LOCO foi aplicado iterativamente no anel de armazenamento, onde foi possível reduzir os erros entre a matriz resposta medida e a matriz nominal por um décimo dos valores iniciais. Medidas foram realizadas com feixe de elétrons no anel a fim de comprovar independentemente os efeitos positivos das correções aplicadas: as funções óticas medidas foram melhores ajustadas aos valores nominais, recuperando parcialmente a simetria da máquina, o acoplamento bétatron global foi praticamente eliminado, houve aumentos consideráveis na abertura dinâmica horizontal, bem como na eficiência de injeção. Os efeitos dos erros de alinhamento e distorções de órbita na ótica e acoplamento do Sirius foram estudados e verificou-se que estes erros contribuem significativamente para os parâmetros do anel. Além disso, foi concluído que o nível das correções obtidas neste trabalho está próximo do limite imposto pelas perturbações de ótica geradas pela órbita residual presente no Sirius.
    
    \vspace{\onelineskip}
    \noindent\textbf{Palavras-chaves}: Sirius; LNLS; fonte de luz síncrotron; comissionamento; anel de armazenamento; ótica linear; acoplamento bétatron; matriz resposta de órbita; LOCO; física de aceleradores.
    \vspace{\fill}
\end{otherlanguage*}
\cleardoublepage

% Resumos em Inglês
\begin{resumo}
\renewcommand{\sfdefault}{\rmdefault}
    Sirius is the new 4\ts{th} generation low emittance synchrotron light source of Brazilian Synchrotron Light Laboratory (LNLS), where $\SI{3}{\giga\electronvolt}$ electrons are kept in stable conditions in ultra-high vacuum along a $\SI{518}{\meter}$ storage ring under the action of electromagnetic fields. The orbit response matrix due to variations in localized dipolar fields can be used to adjust the linear optics and coupling terms in the magnetic lattice from a model, using the method called Linear Optics from Closed Orbits (LOCO). In this work, the LOCO method was studied and implemented in the Sirius storage ring, in order to calibrate and correct linear optics and coupling during commissioning. Several tests were performed with the implemented code using simulated and measured data, obtaining results that verified the method's robustness. The minimization algorithm choice and the inclusion of constraints in gradient variations on quadrupoles were important factors for the method application on Sirius. The LOCO fitting was iteratively applied on the storage ring, where it was possible to reduce the errors between the measured response matrix and the nominal matrix by one-tenth of the initial values. Measurements were performed with the electron beam on the ring to independently verify the positive effects of applied corrections: the measured lattice functions were better adjusted to the nominal values, partially restoring the machine symmetry, the global betatron coupling was practically eliminated, there were considerable increases in the horizontal dynamic aperture as well as injection efficiency. The effects of orbit distortions on Sirius optics and coupling were studied and it was verified that these errors contribute significantly to the storage ring parameters. Besides, it was concluded that the level of corrections obtained in this work is close to the limit imposed by the disturbances generated by the residual orbit present in Sirius.
    
    \vspace{\onelineskip}
    \noindent\textbf{Keywords}: Sirius; LNLS; synchrotron light source; commissioning; storage ring; linear optics; betatron coupling; orbit response matrix; LOCO; accelerator physics.
    \vspace{\fill}
\end{resumo}

% Lista de ilustrações
\pdfbookmark[0]{\listfigurename}{lof}
\renewcommand{\sfdefault}{\rmdefault}
\listoffigures*
\cleardoublepage


% Lista de tabelas
\pdfbookmark[0]{\listtablename}{lot}
\renewcommand{\sfdefault}{\rmdefault}
\listoftables*
\cleardoublepage

% Lista de Acronimos e Abreviações
% \renewcommand{\nomname}{Acronyms}
% \pdfbookmark[0]{\nomname}{las}
% \printnomenclature
\renewcommand{\sfdefault}{\rmdefault}
\printglossaries
\cleardoublepage


% Sumário
\pdfbookmark[0]{\contentsname}{toc}
\renewcommand{\sfdefault}{\rmdefault}
\tableofcontents*
% \tableofcontents*
\cleardoublepage
