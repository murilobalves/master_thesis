\newcommand{\udefint}[2]{\int\!\!\text{d}#1 #2}                 % Integral indefinida
\newcommand{\udefoint}[2]{\oint\!\text{d}#1 #2}               % Integral fechada
\newcommand{\defint}[4]{\int_{#3}^{#4}\!\!\text{d}#1 #2}       % Integral definida
\newcommand{\infint}[2]{\defint{#1}{#2}{-\infty}{\infty}}      % Integra definida infinito
\newcommand{\dertot}[3][{}]{\frac{\mathrm{d}^{#1}#2}{\mathrm{d} #3^{#1}}} % Derivada total
\newcommand{\derpar}[3][{}]{\frac{\partial^{#1}#2}{\partial #3^{#1}}}     % Derivada parcial
\newcommand{\average}[2][{}]{\left\langle #2 \right\rangle_{#1}}  % insert brackets for averaged
\newcommand{\poison}[1]{\left\{ #1 \right\}}    % poison brackets
\newcommand{\fof}[1]{\left(#1\right)} % for easily adjusting parentesis size in expressions of the type f(a,b,c)
% \newcommand{\vect}[1]{\overrightarrow{\boldsymbol{#1}}}
\newcommand{\vect}[1]{\boldsymbol{#1}}
\newcommand{\versor}[1]{\boldsymbol{\hat#1}}
% \newcommand{\tensor}[1]{\overleftrightarrow{\boldsymbol{#1}}} % Tensor
\newcommand{\tensor}[1]{\mathcal{#1}} % Tensor
\newcommand{\fourier}[1]{\tilde{#1}}  % representation of the Fourier Transform
\newcommand{\real}[1]{\Re\left\{#1\right\}}
\newcommand{\imag}[1]{\Im\left\{#1\right\}}
\newcommand{\engw}[1]{\emph{#1}}        % Palavra em Língua Inglesa

 \newcommand{\mc}[3]{\multicolumn{#1}{#2}{#3}}  %multicolumn short cut in tables
 \newcommand{\mr}[3]{\multirow{#1}{#2}{#3}}     %multirow short cut in tables

\newcommand{\pFq}[4][1]{{\,}_{#1}\!\text{F}_{1}\fof{#2;#3;#4}}  %hypergeometric function
\newcolumntype{d}[1]{D{.}{.}{#1} }
