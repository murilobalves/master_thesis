\chapter{The \gls{loco} Algorithm}

\gls{loco}, a shortening for Linear Optics from Closed Orbits, is an algorithm which the main objective is to calibrate the accelerator model in order to reproduce the measured orbit response matrix in the real machine. Once this correspondence is achieved, it is considered that the calibrated model is a representation of the real machine in terms of magnetic lattice. Therefore, one can access the information from the real accelerator by analysing the model. The principal information that is studied in this process is related to linear optics functions, allowing for its disturbances detection and, more importantly, to determine its corrections, pushing the measured parameters to the nominal ones.

This method has been applied to several synchrotron light sources over the years and has been proven to be efficient both to detect optics perturbations and to correct the machine linear optics. Besides that, with the realization of 4\ts{th} generation light sources, which uses innovative and very compact magnetic lattices and optics, some details and subtleties of LOCO should be revisited to successfully apply the algorithm in modern machines such as Sirius storage ring. This chapter is dedicated to present the LOCO algorithm and to discuss the aforementioned details.

\section{Orbit response matrix analysis}

If a $j$-th steering magnet strength is locally varied by the amount $\Delta \theta_j$, the electron closed orbit will be distorted. The horizontal and vertical distortions ($\Delta x$ and $\Delta y$) can be measured by the \gls{bpm}s. For the $i$-th \gls{bpm}, the following quantities can be calculated in this process:

\begin{align}
    M^{uv}_{ij} &= \dfrac{\Delta u_i}{\Delta \theta_j^v}
\end{align}

If the steering magnet is horizontal, $v=x$ and if it is vertical, $v=y$. For each corrector varied, one can measure the corresponding positions variations given by every \gls{bpm}, horizontally $u=x$ and vertically $u=y$. Then this values can be cast in an array, called orbit response matrix. The general organization of the elements in this matrix is the following:

\begin{equation}
    \mathbf{M} = \begin{bmatrix}
    \mathbf{M}^{xx} & \mathbf{M}^{xy} \\
    \mathbf{M}^{yx} & \mathbf{M}^{yy} 
\end{bmatrix}
\end{equation}

\begin{align}
M_{ij}^{\mu\mu} = \dfrac{\sqrt{\beta_{\mu}(s_i)\beta_{\mu}(s_j)}}{2\sin\left(\pi\nu_{\mu}\right)}\cos\left[ |\varphi_{\mu}(s_i) - \varphi_{\mu}(s_j)| - \pi\nu_{\mu} \right]
\label{matrix}
\end{align}


\begin{align}
    u_i\left({\theta_j+\Delta\theta_j}\right) = u_i\left(\theta_j\right) + \dfrac{\partial u_i}{\partial \theta_j} \Delta \theta_j
\end{align}

\begin{equation}
    M^{u u}_{ij} = \dfrac{\Delta u_i (+\Delta \theta_j/2) - \Delta u_i (-\Delta \theta_j/2)}{\Delta \theta_j}.
    \label{medida}
\end{equation}

$M_{ij} = \dfrac{\partial u_i}{\partial \theta_j} + O(3)$.





\section{Minimization problem}

\section{Functionalities}
\subsection{Finding errors}
\subsection{Optics correction}
\subsection{Coupling correction}

\section{Degeneracies}

\section{Constraints and weights}

